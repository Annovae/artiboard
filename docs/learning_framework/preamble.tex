\usepackage[top=1.25in,bottom=1.3in,left=1.25in,right=1.25in]{geometry}
\usepackage{fp}
%\usepackage[body={6.0in, 8.2in},left=1.25in,right=1.25in]{geometry}  
\usepackage{amsmath,amssymb}             % AMS Math 
\usepackage{pstricks, pst-node,pst-tree}
\usepackage{ntabbing}
\usepackage{subfigure}
% a list with labelled items
\newenvironment{mylist}
{\begin{list}
	{}
	{
		\setlength{\topsep}{0.5em}
		\setlength{\partopsep}{0pt}
		\setlength{\parskip}{0pt}
		\setlength{\parsep}{0pt}
		\setlength{\itemsep}{\topsep}	% space between items
		\setlength{\labelwidth}{1em}
		\setlength{\labelsep}{5pt}          % between label and text
		\setlength{\itemindent}{\labelsep}    % indent start of item paragraph
		\setlength{\listparindent}{1em}
		\setlength{\leftmargin}{1.5em}    % all items
		\setlength{\rightmargin}{0pt}    % all items
		\renewcommand\makelabel[1]{$*$ \myterm{##1}.}
	}
}
{\end{list}}

% a list without labelled items
\newenvironment{mydesclist}
{\begin{list}
	{}
	{
		\setlength{\topsep}{0.5em}
		\setlength{\partopsep}{0pt}
		\setlength{\parskip}{0pt}
		\setlength{\parsep}{0pt}
		\setlength{\itemsep}{\topsep}	% space between items
		\setlength{\labelwidth}{1em}
		\setlength{\labelsep}{0pt}          % between label and text
		\setlength{\itemindent}{\labelsep}    % indent start of item paragraph
		\setlength{\listparindent}{1em}
		\setlength{\leftmargin}{1.5em}    % all items
		\setlength{\rightmargin}{0pt}    % all items
		\renewcommand\makelabel[1]{$*$ \sl{##1}}
	}
}
{\end{list}}

%pseudo code
\newenvironment{code} 
	{\begin{sffamily}
	\begin{tabbing}} 
	{\end{tabbing}
	\end{sffamily} }

%references
\newcommand{\reftab}[1]{Table \ref{tab:#1}}
\newcommand{\refeq}[1]{Equation \ref{eq:#1}}
\newcommand{\refsec}[1]{Section \ref{sec:#1}}
\newcommand{\reffig}[1]{Figure \ref{fig:#1}}
\newcommand{\refalg}[1]{Figure \ref{alg:#1}}
\newcommand{\refline}[1]{line \ref{line:#1}}
\newcommand{\refdef}[1]{Definition \ref{def:#1}}

% special words and phrases
\newcommand{\etal} {{\it et. al.} }
\newcommand{\chapterintro} {\it \small}
\newcommand{\mytdist} {{\sl t}-distribution}
\newcommand{\myFdist} {{\sl F}-distribution}
\newcommand{\myatdist} {{\sl *t}-distribution}
\newcommand{\myFvalue} {{\sl F}-value}
\newcommand{\mytvalue} {{\sl t}-value}
\newcommand{\svar}[1]{s^2_{\mbox{\tiny #1}}} % sample variance
\newcommand{\var}[1]{\sigma_{\mbox{\tiny #1}}^{2}}  % variance
\newcommand{\mean}[1]{\overline{X}_{\mbox{\tiny #1}}}  % sample mean is X
\newcommand{\Fdist}[3]{{\it F}^{(#1)}_{#2,#3}} % F-Distribution
\newcommand{\tdist}[2]{{\it t}^{(#1)}_{#2}} % t-Distribution
\newcommand{\header}[1]{\subsubsection{#1}}
\newcommand{\prob}[1]{\mbox{Pr}(#1)} % probability
\newcommand{\mygame}[1]{\textsc{#1}}  % commands used for game names
\newcommand{\myprogram}[1]{\textbf{\it #1}} % the name of a program 
\newcommand{\myset}[1]{{\bf #1}} % a symbol representing a set
\newcommand{\myrel}[1]{{\bf #1}} % a symbol representing a relation
\newcommand{\mydom}[1]{\mathbb{#1}} % a defined set
\newcommand{\mysetof}[1]{\{e \; | \; e \in #1\}} % a symbol 
\newcommand{\mytabheader}[1]{\bf #1} % heading in a tabular
\newcommand{\myheading}[1]{\par\vspace{10pt}\noindent{\bf #1}
\newline\par}	% essentially a  \subsubsubsection
\newcommand{\merge}{\oplus}
\newcommand{\myfrac}[2] { \left(\begin{array}{c} #1 \\#2\end{array}\hspace{-5pt}\right) }
% Symbols used
\newcommand{\pbest}[1]{\mbox{\it pbest}(#1)}
\newcommand{\pbestt}[1]{\mbox{\it pbest}_t(#1)}
\newcommand{\gbest}[1]{\mbox{\it gbest}(#1)}
\newcommand{\lbest}[1]{\mbox{\it lbest}(#1)}
\newcommand{\lbestt}[1]{\mbox{\it lbest}_t(#1)}
\newcommand{\neighbours}[1]{\myset N_{\vec{#1}}}
	% component of vector
\newcommand{\comp}[2]{{#1}[#2]}   
	% velocity of particle
\newcommand{\vel}[1]{v(#1)}   
\newcommand{\velcomp}[2]{v(#1)[#2]}   
\newcommand{\velepoch}[2]{v_{#2}(#1)}   
	% component of velocity of particle
\newcommand{\velepochcomp}[3]{v_{#2}(#1)[#3]}    
\newcommand{\mywin}{{\mathcal W}}
\newcommand{\mylose}{{\mathcal L}}
\newcommand{\mydraw}{{\mathcal D}}
\newcommand{\myor}{\vee}
\newcommand{\myand}{\;\wedge\;}
\newcommand{\myunion}{\cup}
\newcommand{\myintersect}{\cap}
\newcommand{\myst}{\; | \;} % such that
\newcommand{\mysquare}[2]{(#1,#2)} % file, rank
\newcommand{\mysquares}{\mydom{S}}
\newcommand{\mypieces}{\mydom{O}}
\newcommand{\mycardinalities}{\mydom{C}}
\newcommand{\myatoms}{\mydom{A}}
\newcommand{\mypositions}{\mydom{P}}
\newcommand{\myplacements}{\myrel{P}}
\newcommand{\myints}{\mathbb Z}
\newcommand{\mynaturals}{\mathbb N}
\newcommand{\mytrue}{\mbox{\it true}}
\newcommand{\myfalse}{\mbox{\it false}}

% help with equations
\newcommand{\mywhere}{\noindent where}
\newcommand{\myarg}{\newline\indent\indent}
\newcommand{\myreal}{{\mathbb R}}
\newcommand{\mymap}{\rightarrow}
\newcommand{\mywff}{\textit{wff}}
\newcommand{\mywffs}{{\it wff}s}

\newcommand \myprint [1] {\FPround \fpr {#1} 6 \FPprint \fpr}
\newcommand \myeval [1] {\FPeval \val {#1} \FPround \rval \val 6 \FPprint \rval}

% New Environments
%\newtheorem{mydef}{Definition}[section]
%\newcounter{mydef}[chapter]
% algorithms must be encapsulated in figures
\newenvironment{algorithm}[3] 
% the first parameters is the name
% the second parameter describes the input
% the third parameter describes the output
% the body of the environment contains the pseudo code  lines are terminated with \\ and \> is used to indent
{
\small
\begin{tabular} [t] {rp{60ex}}
\hline
\\
	{\it Algorithm} & #1 \\
	{\it Input} & #2  \\
	{\it Output} & #3 \\
\end{tabular}
%	\begin{sffamily}
	%\tt		% adding tt looks good but causes latex font warnings
	\small
	\begin{ntabbing}

	\reset
	123\=123\=123\=123\=123\=123\=\kill
	
}{
	\end{ntabbing}
%	\end{sffamily}
}

% Name of a new parameter function
\newcommand{\paramfun} {\pi_{\arabic{equation}}}
% Commands to draw a board
\newcommand{\mycenter} [1] {\parbox{1.1em}{\centering{#1}}}
\newcommand{\myrowa} [5] {\hline \textbf{#1}& \mycenter{#2}&
&\mycenter{#3}&&\mycenter{#4}& &\mycenter{#5}& }
\newcommand{\myrowb} [5] {\hline \textbf{#1}& &\mycenter{#2}& &\mycenter{#3}&
&\mycenter{#4}& &\mycenter{#5}}
\newcommand{\myrefboard} {Board diagram \arabic{section}.\arabic{board}}
\newcommand{\myrefstepboard} {\refstepcounter{board} \myrefboard}
\newcommand{\myboard} [8]
{
\begin{center}
\small
\begin{tabular}[!ht] {c|c|c|c|c|c|c|c|c|}
\multicolumn{9}{c}{Active (o)} \\
 \myrowb {8}   #1 \\ 
 \myrowa {7}   #2 \\ 
 \myrowb {6}   #3 \\ 
 \myrowa {5}   #4 \\ 
 \myrowb {4}   #5 \\ 
 \myrowa {3}   #6 \\ 
 \myrowb {2}   #7 \\
 \myrowa {1}   #8 \\
\hline
  &\textbf{1}&\textbf{2}& \textbf{3} & \textbf{4} & \textbf{5} & \textbf{6}  & \textbf{7} & \textbf{8} \\
\multicolumn{9}{c}{Passive (x)} \\
\end{tabular}
\end{center}
\begin{center}
\myrefboard \\
\end{center}
}

\newcounter {board} [section]

% Usage: \
%		statsTable  {label} {caption} {body}
\newcommand{\statsTable} [4]
{
	\begin{table} [h!]
	\small
	\centering
	\begin{tabular}{|c|c|c|c|c|}
	\hline
	S & $\mean S$ & $\svar S$ & Max & Min \\
	\hline
	#3
	\hline
	\end{tabular}
	\caption{#2}
	\label{tab:#1}
	\end{table}
}